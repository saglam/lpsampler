\documentclass[11pt,letterpaper]{article}

\usepackage{amsmath, amsthm, amssymb}
\usepackage[hmargin=3.4cm,vmargin=3.2cm]{geometry}
\usepackage{authblk}
\usepackage[runin]{abstract}
\usepackage{sectsty}
\usepackage[font=small,labelfont=small]{caption}
\usepackage[usenames]{xcolor}
\definecolor{Blue}{RGB}{0, 38, 105}
\definecolor{Green}{RGB}{49, 127, 52}
\usepackage{hyperref}
\hypersetup{
    colorlinks,
    citecolor=Green,
    filecolor=Blue,
    linkcolor=Blue,
    urlcolor=Blue
}

\newtheorem{definition}{Definition}
\newtheorem{lemma}{Lemma}
\newtheorem{theorem}{Theorem}
\newtheorem{proposition}[theorem]{Proposition}
\newtheorem{conjecture}{Conjecture}
\theoremstyle{remark}
\newtheorem*{remark}{Remark}

\title{\Large\bf Tight Bounds for $L_p$ Samplers, Finding Duplicates in
  Streams, and Related Problems}
\renewcommand\Authfont{\normalsize}
\renewcommand\Affilfont{\small}
\author[1]{Hossein Jowhari}
\author[1]{Mert Sa\u{g}lam}
\affil[1]{Simon Fraser University, Burnaby, Canada}
\author[1,2]{G\'abor Tardos}
\affil[2]{R\'enyi Institute of Mathematics, Budapest, Hungary}
\date{}

\DeclareMathOperator*{\E}{\mathbb{E}}
\DeclareMathOperator{\R}{R}
\DeclareMathOperator{\err}{Err}
\DeclareMathOperator{\UR}{UR}
\DeclareMathOperator{\URn}{UR^n}
\DeclareMathOperator{\polylog}{polylog}
\DeclareMathOperator{\poly}{poly}
\DeclareMathOperator*{\Ent}{H}
\DeclareMathOperator*{\BEnt}{H_2}
\DeclareMathOperator*{\I}{I}

\newcommand{\dmid}{\,\|\,}
\newcommand{\emid}{\,|\,}

\def\equationautorefname{Equation}
\def\sectionautorefname{Section}
\def\subsectionautorefname{Section}
\def\lemmaautorefname{Lemma}
\def\conjectureautorefname{Conjecture}

\begin{document}
\maketitle

% !TeX root = main.tex

\setlength{\abstitleskip}{-\absparindent}
\vspace{-7.5mm}
\begin{abstract}
In this paper, we present near-optimal space bounds for
$L_p$-samplers. Given a stream of updates (additions and
subtraction) to the coordinates of an underlying vector 
$x \in\mathbb R^n$, a perfect $L_p$ sampler outputs the
$i$-th coordinate with probability $|x_i|^p/\|x\|_p^p$ while an approximate
$L_p$ sampler, roughly speaking, outputs the same coordiante with a probability witin the range
$(1\pm \epsilon)|x_i|^p/\|x\|_p^p$.
In SODA 2010, Monemizadeh and Woodruff showed polylog space
upper bounds for approximate $L_p$-samplers and demonstrated
various applications of them.
%
A year later, Andoni, Krauthgamer and Onak improved the
upper bounds and gave a $O(\epsilon^{-p}\log^3 n)$ space
$\epsilon$ relative error and constant failure rate
$L_p$-sampler for $p\in [1,2]$.
%
In this work, we give another such algorithm requiring only
$O(\epsilon^{-p}\log^2n)$ space for $p \in (1,2)$. For 
$p \in (0,1)$, our space bound is $O(\epsilon^{-1}\log^2n)$, 
while for the $p=1$ case we have an 
$O(\log(1/\epsilon)\epsilon^{-1}\log^2n)$ space algorithm.
We also give a $O(\log^2 n)$ bits zero relative error
$L_0$-sampler, improving the $O(\log^3 n)$ bits algorithm due
to Frahling, Indyk and Sohler.

%This considerably improves the previous bound due to
%Monemizadeh et al.

As an application of our samplers, we give better upper bounds
for the problem of finding duplicates in data streams. In case
the length of the stream is longer than the
alphabet size, $L_1$ sampling gives us an $O(\log^2 n)$ space
algorithm, thus improving the previous $O(\log^3 n)$ bound due
to Gopalan and Radhakrishnan.

In the second part of our work, we prove an $\Omega(\log^2 n)$
lower bound for sampling from 0, $\pm1$ vectors 
(in this special case, the parameter $p$ is not relevant for 
$L_p$ sampling). This matches the space of our sampling 
algorithms for constant $\epsilon>0$. We also prove tight space 
lower bounds for the finding duplicates and heavy hitters 
problems. We obtain these lower bounds using reductions from 
the communication complexity problem Augmented Indexing.
\end{abstract}

  
% !TeX root = main.tex

\section{Introduction}
\label{sec:intro}
Sampling has become an indispensable tool in analysing massive data sets,
and particularly in processing data streams. In the past decade, 
several sampling techniques have been proposed and studied for the data stream
model \cite{BabcockDM02,DuffieldLT07,BravermanOZ09,CormodeMYZ10,MonemizadehW10,AndoniKO10}.
In this work, we study {\em $L_p$-samplers}, a new variant of
space efficient samplers for data streams that
was introduced by Monemizadeh and
Woodruff in \cite{MonemizadehW10}. 
 Roughly speaking, given a stream of updates (additions and subtraction) 
to the coordinates of an underlying vector $x \in \mathbb R^n$,
an $L_p$-sampler processes the stream and 
outputs a sample coordinate of $x$ where the
$i$-th coordinate is picked with probability proportional to
$|x_i|^p$.

In \cite{MonemizadehW10}, it was observed that $L_p$-samplers lead to 
alternative algorithms for many known streaming problems, including  
heavy hitters and frequency moment estimation. Here in this paper, we
focus on a specific application, namely finding duplicates in long streams;
although our $L_p$ samplers  work and often give better space performance
for all applications listed in \cite{MonemizadehW10}.
We refer the reader to \cite{MonemizadehW10} and \cite{AndoniKO10} 
for further applications of $L_p$-samplers.

%$L_p$ samplers handle both insertions and deletions of data
 %and are shown to be very useful in designing streaming algorithms in general.\

Observe that we allow both negative and positive updates to the coordinates of the underlying vector.
In the case where only positive updates are allowed and $p=1$, the problem is well understood.
%Typically and in this paper, the input stream
  %represents as a sequence of updates of
  %the form $(i,u)$, with $i \in [n]=\{1,\ldots,n\}$, that indicates
  %an addition of $u \in [-M,M]$ to $i$-th coordinate of
   %an initially zero vector $x \in \mathbb R^{n}$. 
   %A basic problem in the processing of update
    %streams is to maintain
 % samples from $x$, where $i$-th coordinate is picked with 
  %probability proportional to $\frac{|x_i|}{\|x\|_1}$.
   %Sampling in the case of only positive updates is well understood.  
    For instance the classical reservoir sampling \cite{Knuth69} from the 60's
    (attributed to Alan G.~Waterman) gives a simple 
    solution as follows. Given a pair $(i,u)$, 
    indicating an addition of $u$ to the  $i$-th coordinate of 
    the underlying vector $x$, the sampler having maintained $s$, the sum of the 
    updates seen so far, replaces
     the current sample with $i$ with probability $u/s$, otherwise does nothing 
     and moves to the next update. It is easy to verify that this is a perfect
     $L_1$-sampler and the space usage is only $O(1)$ words.
    
   With the presence of negative updates, sampling becomes a non-trivial problem. In this case,
   it is not clear at all how to maintain samples without keeping track of the updates
    to the individual coordinates. In fact, the question
    regarding the mere existence of such samplers was raised few years ago by Cormode,
    Muthukrishnan, and Rozenbaum in \cite{CormodeMR05}. Last year in SODA 2010, 
    Monemizadeh and Woodruff \cite{MonemizadehW10} answered this question
    affirmatively, however in an approximate sense. Before stating 
     their results we give a formal definition of 
    $L_p$-samplers.

\begin{definition}
Let $x \in \mathbb{R}^n$ be a non-zero vector. For
$p>0$ we call the {\em $L_p$ distribution} corresponding to $x$ the
distribution on $[n]$ that takes $i$ with probability
$$\frac{|x_i|^p}{\|x\|_p^p},$$
with $\|x\|_p=(\sum_{i=1}^n|x_i|^p)^{1/p}$. For $p=0$,
the $L_0$ distribution
corresponding to $x$ is the uniform distribution over the non-zero coordinates
of $x$.
\end{definition}

We call a streaming algorithm a {\em perfect $L_p$-sampler} if it outputs an
index according to this distribution and fails only if $x$ is the zero
vector. An approximate $L_p$-sampler may fail but the distribution of its
output should be close to the $L_p$ distribution. In particular, we speak of an $\epsilon$ relative
error $L_p$-sampler if, conditioned on no failure, it outputs the index $i$
with probability $(1\pm\epsilon)|x_i|^p/\|x\|_p^p+O(n^{-c})$, where $c$ is an
arbitrary constant. For $p=0$ the corresponding formula is
$(1\pm\epsilon)/k+O(n^{-c})$, where $k$ is the number of non-zero coordinates
in $x$. Unless stated otherwise we assume
that the failure probability is at most $1/2$.

In this definition one can consider $c$ to be 2, but all existing
constructions of $L_p$-samplers work for an arbitrary $c$ with just a
constant factor increase in the space, so we will not specify $c$ in the
following and ignore errors of probability $n^{-c}$.
 
%\paragraph{Previous Works on $L_p$ sampling.} The authors in \cite{MonemizadehW10}, gave
% various upper bounds for $L_p$ samplers. In particular, for
\paragraph{Previous work.} 
A zero relative error $L_0$-sampler which uses $O(\log^3 n)$ bits was shown in \cite{FrahlingIS05}. In \cite{MonemizadehW10}, the authors gave an $\epsilon$ relative error
$L_p$-sampler for $p \in [0,2]$ which uses $\poly(\epsilon^{-1},\log n)$ space. They also showed 
  a 2-pass $O(\polylog n)$ space zero relative error $L_p$-sampler for any $p\in [0,2]$. In addition
  to these, they demonstrated that $L_p$-samplers can be used as a black-box to obtain
   streaming algorithms for other problems such as $L_p$ estimation (for $p >2$), heavy hitters, 
   and cascaded norms \cite{JayramW09}.
   Unfortunately, due to the large exponents in their bounds, the
  $L_p$-samplers given there do not lead to efficient solutions for the aforementioned applications.
  
  Very recently, Andoni, Krauthgamer and Onak in %an arxiv submission %-Mert: I haven't seen people referring to archive in the paper, but if you like it that way lets put it back.
\cite{AndoniKO10} improved the results of \cite{MonemizadehW10} 
 considerably. Through the adaptation of %-Mert: Also, I'm  could not be sure about 'improve the situation'. Do you think ameliorate better?
  a generic and simple method, named {\it precision sampling}, they managed
   to bring down the space upper bounds to $O(\frac1{\epsilon^p}\log^3 n)$ bits
   for $\epsilon$ relative error $L_p$-samplers for $p \in [1,2]$. Roughly speaking, the idea of precision
  sampling is to scale the input vector with random coefficients so that the $i$-th coordinate
  becomes the maximum with probability roughly proportional to $|x_i|^p$. Moreover the maximum (heavy)
   coordinate is found through a small-space heavy hitter algorithm. In more detail,
    the input vector $(x_1,\ldots,x_n)$ is scaled 
   by random coefficients $(t_1^{-1},\ldots,t_n^{-1})$, where each $t_i$ is picked uniformly 
   at random from $[0,1]$. Let $z=(x_1t_1^{-1},\ldots,x_nt_n^{-1})$ be the 
   scaled vector. %Notice that the random coefficients are chosen independent of $x$
   %and from a small probability space. Also s
%-Mert: Here small probability space seemed to be a bit confusing to me.
%-Mert: On the other hand, if we try to explain cutting off the distribution at n^4, this will take too much space. I left it out for now.
%Since
 %  the mapping is linear, additions and subtraction from $x$ can
 %  be handled easily by storing random bits needed. 
 Here the 
   important observation is $\Pr[t_i^{-1} \ge t] =1/t$ 
   and hence, for instance,  by replacing $t$ with 
   $\|x\|_1/|x_i|$, we get $\Pr[|z_i| \ge \|x\|_1] = |x_i|/\|x\|_1$. (In the
   same manner, one can scale $x_i$ by $t_i^{-1/p}$
   instead of $t_i^{-1}$ and get a similar result for general $p$.) %It follows,
  % sacrificing a bit of precision, with this method, we
%  only need to know a constant factor approximation of
It turns out, we only need to we have a constant approximation to
%-Mert: If we settle for a constant approximation of the norm, I think, we don't lose precision. We just need to repeat more.
%-Mert: So i changed this accordingly. 
$\|x\|_1$ and look for a coordinate in $z$ that has reached a
   limit of $\Omega(\|x\|_1)$. On the other hand it is shown that the heaviest coordinate in $z$ has
   a weight of $\Omega(\log^{-1}{n})\|z\|_1$ (with constant probability), and thus
   a small-space heavy hitter computation can be used to find the maximum. In particular, 
   the $L_p$-sampler of \cite{AndoniKO10} adapts the
   popular {\it count-sketch} scheme \cite{CharikarCF04} for this purpose.
   % Upper bounding $\E[t_i^{-1}]$ by 
   %$O(\log n)$ (for a bounded version $t_i$) and assuming the space usage of
   % $O(\phi^{-1}\log^2 n)$ for finding a $\phi$-HH, the final space upper bound achieved in \cite{AndoniKO10} 
   %becomes roughly $O(\frac{1}{\epsilon^p}\log^3 n)$ bits for $p \in [1,2]$.
   \paragraph{Our contributions.} In %the main result of
   this paper,  we  give $L_p$-samplers requiring only
$O(\epsilon^{-p}\log^2n)$ space for $p \in (1,2)$. For $p \in (0,1)$,
our space bound is  $O(\epsilon^{-1}\log^2n)$, while for the $p=1$ case we have
an $O(\log(1/\epsilon)\epsilon^{-1}\log^2n)$ space algorithm.
   %improve the space usage of $L_p$ samplers to $O(\epsilon^{-\max\{1,p\}}\log^2 n)$ bits
   %for $p \in (0,2)$ with an extra $\log(1/\epsilon)$
   %factor appearing in our bounds in the $p=1$ case (Theorem \ref{thm:sampler}).
    In essence, our sampler follows the basic structure of the
     precision sampling method explained above. % namely scaling each coordinate by a random real and finding 
%      the heavy hitters. 
      However compared to \cite{AndoniKO10}, we
      do a sharper analysis of the error terms in the count-sketch, and through
      %some modifications,
    additional ideas, we manage to get rid of a log factor
      and preserve the previous dependence on $\epsilon$. 
      Roughly speaking, we use the fact that 
      the error term in the count-sketch is bounded by the $L_2$ 
      norm of the tail distribution of $z$ (the heavy coordinates do not contribute). On
      the other hand, taking 
     the distribution of the random coefficients into account, we
     bound this by $O(\|x\|_p)$, which enables us to save a 
     log factor. Additionally, to preserve
      the dependence on $\epsilon$, we have to use a slightly more 
     powerful source of randomness for choosing the scaling factors
     (in contrast with the pairwise-independence of \cite{AndoniKO10}), and 
     take care of some subtle issues regarding the conditioning on
       the error terms which are not handled in the previous work 
       (Lemma \ref{abort}).
  
% For $p$ near zero, the method of precision sampling
  % becomes intractable. The reason being $x_i$
   %is multiplied by $t_i^{-1/p}$ which clearly rules out $p=0$. 
   %For this special case, we present an algorithm using a 
   %completely different approach.
     
   As $p$ approaches zero, precision sampling becomes very inefficient, as
  the random coefficients $t_i^{-1/p}$ tend to infinity. For the $p=0$ case,
 we present a completely different algorithm. 
   Briefly, our $L_0$-sampler tries to 
   detect a non-zero
   coordinate by picking random subsets of $[n]$.
   The non-zero coordinates are found
   by an exact sparse recovery procedure and Nisan's PRG \cite{Nisan} 
   is applied to decrease the randomness involved.
   Our $O(\log^2 n)$ space bound compares favorably to the previous algorithms, which use respectively $O(\log^3 n)$ space \cite{FrahlingIS05} and $\poly(\log n,\epsilon^{-1})$  space \cite{MonemizadehW10} (the latter one gives only $\epsilon$-relative error sampling).
%In comparison with the previous $\epsilon$ relative error $L_0$-sampler
  % due to Monemizadeh et al., our algorithm gives
    %zero-relative error guarantees and the space bound is $O(\log^2 n)$ bits.

    In \autoref{sec:lb}, we prove that sampling from 0, $\pm1$
   vectors requires $\Omega(\log^2 n)$ space, by a reduction from 
   the communication complexity problem augmented indexing. 
   In this special case $p$ is not relevant for $L_p$-sampling, hence 
   this shows that our $L_0$-sampling algorithm uses the optimal space up to constant 
   factors, and our $L_p$-sampler for $p\in(0,2)$ has the optimal space (up to constant factors) for $\epsilon>0$ a constant. 
%To prove the optimality of our upper bounds for $L_p$ samplers, 
%we study the streaming and communication complexity of {\it finding
%duplicates} in long streams.Formally, 

Given a stream of length $n+1$ over the alphabet $[n]$, finding
duplicates problem asks to output some $a\in[n]$ that has appeared
at least twice in the stream. Observe that by the pigeon-hole principle, 
such $a$ always exists. Prior to our work, the best upper bound
for finding duplicates was due to  Gopalan
and Radhakrishnan \cite{GopalanJaikumar}, who gave a one-pass $O(\log^3
n)$ bits randomized algorithm with constant failure rate.
Here we settle the one-pass complexity of this problem by 
giving an $O(\log^2 n)$ space algorithm via a direct application of our $L_1$
sampler, and by giving an $\Omega(\log^2 n)$ lower bound afterwards. Combined with a sparse recovery procedure,
our solution also generalizes to a near-optimal $O(\log^2n+s\log n)$ space
algorithm for finding duplicates in streams of length $n-s$, improving on the $O(s\log^3 n)$ result of \cite{GopalanJaikumar}.
  

%Next we obtain a $\Omega(\log^2n)$ space lower bound
%for 1-pass algorithms that finds 
%duplicates (Theorem \ref{thm:duplb}). This problem,
%in the 2-player communication setting, happens to be
%a well-studied problem in the sphere of 
% deterministic protocols \cite{TardosZwick,KarchmerWigderson}. Known
%as {\it universal relations}, in this 2-player game, each of Alice and Bob
% gets a vector $x,y\in\{0,1\}^n$ respectively with the promise that $x\neq y$.
%The players send each other messages and the last player to receive
%a message must output an index $i\in [n]$ such that $x_i\neq y_i$. 
%We denote this problem by $\UR_n$. Now 
%if we interpret $x_i$ and $y_i$ as respectively a positive 
%and negative increment to the $i$-th coordinate of 
%an initially vector $z$, namely $z=x-y$, any constant relative
%error $L_p$ run over $z$  almost certainly outputs an index $i\in [n]$ where Alice and Bob differ.
%Consequently any lower bound for $\UR_n$ will lower bound the space complexity
%of $L_p$ samplers. To derive a lower bound for $\UR_n$, we use a reduction from the well-known
%{\it augmented indexing} (see Section \ref{sec:lb} for a definition).  

%Finally, in addition to a
%lower bounds for $L_p$ samplers, through a similar reduction from augmented indexing,
% we obtain a tight lower bound for
%the closely related problem of finding heavy hitters in update streams (also
%in the strict turnstile model). 
%According to our knowledge, this is the first result of this kind for heavy hitters
%under negative updates.   

Finally, we prove lower bounds for the problem of finding heavy hitters in update streams, 
which is closely related to the $L_p$-sampling problem. This lower bound is also obtained 
by a reduction from the augment indexing and proves that any $L_p$ heavy 
hitters algorithm (defined in Section \ref{sec:hh}) must use 
$\Omega(\frac{1}{\phi^p}\log^2 n)$ space, even in the strict turnstile model. 
Our lower bound essentially matches the known upper bounds \cite{CormodeM05,CharikarCF04,KaneNPW} which work in the general update model. 

\paragraph{Related work.} In \cite{BabcockDM02,BravermanOZ09}, the authors have 
studied sampling from sliding windows, and the recent paper
of Cormode et al.\ \cite{CormodeMYZ10} generalizes the classical reservoir sampling 
to distributed streams. These works only support insertion streams.
 The basic idea of random scaling used in \cite{AndoniKO10} and in our paper has appeared earlier in
 the priority sampling technique \cite{DuffieldLT07,CohenDKLT09}, where the focus is
 to estimate the weight of certain subsets of a vector, defined by a sequence of positive updates.
 
Finding duplicates in streams was first considered 
in the context of detecting fraud in click streams \cite{MetwallyAA05}. 
Muthukrishnan in \cite{Muthukrishnan}
asked whether this problem can be solved in 
$O(\polylog n)$ space using a constant number of passes. In \cite{Tarui}, Tarui showed
that any $k$-pass deterministic algorithm must use $\Omega(n^{\frac1{2k-1}})$
space. 
%Finally, Gopalan and Radhakrishnan gave a randomized $O(\log^3 n)$ space algorithm, answering the question posed by Muthukrishnan.
 
Heavy hitter algorithms have been studied extensively. 
The work of Berinde et al.\
\cite{BerindeCIS09} gives tight lower bounds for heavy hitters under 
 insertion-only streams. We are not aware of similar works on
general update streams, although the recent works of \cite{BaIPW10, WoodruffJ11},
 where the authors show lower bounds for respectively approximate
sparse recovery, and Johnson-Lindenstrauss transforms (via augmented indexing) is closely related.    
    

\paragraph{Notation.}
%We write $[n]$ for the set $\{1,\ldots,n\}$.
%As we explained above, the output of the typical streaming problem we
%consider depends on a real vector $x\in\mathbb R^n$,
%initially zero and then modified by a stream of updates. An update of the form
%$(i,u)$ with $i\in[n]$ and $u\in\mathbb R$ adds $u$ to the
%coordinate $x_i$ of $x$ (leaving the other coordinates unchanged).
%In the {\em strict turnstile model}
%we are guaranteed that all coordinates of $x$ are non-negative at the end
%(although negative updates are still allowed), in the general model such
%guarantee does not exist. 
We write $[n]$ for the set $\{1,\ldots,n\}$.
An update stream is a sequence of tuples $(i,u)$, where $i\in [n]$ 
and $u\in\mathbb R$. The stream of updates implicitly define an $n$-dimensional
vector $x\in\mathbb R^n$ as follows. Initially, $x$ is the zero vector. An update of the form
$(i,u)$ adds $u$ to the coordinate $x_i$ of $x$ (leaving the other coordinates unchanged).
In the {\em strict turnstile model}
we are guaranteed that all coordinates of $x$ are non-negative at the end
of the stream
(although negative updates are still allowed), in the general model such
guarantee does not exist. Our
algorithms (like most other algorithms in the literature) work by maintaining
a linear sketch $L:\mathbb R^n\to\mathbb R^m$.
When computing the space requirement of such a
streaming algorithm, we assume all the updates are integers ($u\in\mathbb Z$)
and the coordinates of the vector $x$ throughout the stream remain bounded by
some value $M=\poly(n)$. We
make sure that the matrix of $L$ has also polynomially bounded
integer entries, this way maintaining $L(x)$ requires updating $m$ integer
counters and requires $O(m\log n)$ bits with fast
update time (especially since the matrices we consider are sparse). This
discretization step is standard and thus we omit most details.

In the standard model for randomized streaming algorithms the random bits used
(to generate the random linear map $L$, for example) are part of the space
bound. In contrast, our lower bounds do not make any assumption on the working of the
streaming algorithm and allow for the {\em random oracle model}, where the
algorithm is allowed free access to a random string at any time. All lower
bounds are proved through reductions from communication problems.

We say an event happens with {\em low probability} if the probability can
be made less than $n^{-c}$. Here $c>0$ is an arbitrary constant, for example
one can set $c=2$. The actual value of $c$ has limited effect on the space
of our algorithm: it changes only the unspecified constants hidden in the $O$
notation. We will routinely ignore low probability events, sometime even
$O(n)$ of them, which is okay as we leave $c$ unspecified.
Events complementary to low probability events are referred to as {\em high
probability} events.

For $0\le m\le n$ we call the vector $x\in\mathbb R^n$ {\em $m$-sparse} if
all but at most $m$ coordinates of $x$ are zero. We define
$\err_2^m(x)=\min\|x-\hat x\|_2$, where $\hat x\in\mathbb R^n$ ranges over all
the $m$-sparse vectors.

%\paragraph{Organization.} In Section 2 we present our $L_p$ samplers, in
%Section 3 we apply it to the duplicates problem, while in the last section we
%prove our lower bounds. For completeness we present a simple modification of
%the heavy hitter algorithm of \cite{KaneNPW} in the Appendix.

%-Mert: I removed this for now as (1) space considerations (2) we gave some section numbers in Section 'Our Contribution', which will hopefully guide the reader




% !TeX root = lpsampler.tex
\section{The $L_p$ Sampler}
\label{sec:lpsamp}
In this section, we present our $L_p$ sampler algorithm. In the following,
we assume  $p \in (0,2)$. This particular method does not seem to be
applicable for the $p=2$ case and we know of no $O(\log^2 n)$ space
$L_2$-sampling algorithm. We treat the $p=0$ case separately later.

We start by stating the properties of the two streaming algorithms we are
going to use. Both are based on maintaining $L(x)$
for a well chosen random linear map
$L:\mathbb R^n\to\mathbb R^{n'}$ with $n'<n$.

The {\em count-sketch} algorithm \cite{CharikarCF2004} is so simple we cannot resist the
temptation to define it here. For parameter $m$, the count-sketch algorithm
works as follows. It selects independent samples $h_j:[n]\to[6m]$ and
$g_j:[n]\to\{1,-1\}$ from pairwise independent uniform hash families for
$j\in[l]$ and $l=O(\log n)$. It computes the following linear function of $x$
for $j\in[l]$ and $k\in[6m]$:
$y_{k,j}=\sum_{i\in[n],h_j(i)=k}g_j(i)x_i$. Finally it outputs
$x^*\in\mathbb R^n$ as an approximation of $x$ with
$x^*_i=\hbox{median}(g_j(i)y_{h(i),j}:j\in[l])$ for $i\in[n]$.

The performance guarantee of the count-sketch algorithm is as follows.
(For a compact proof see a recent survey by Gilbert and Indyk \cite{GilbertI2010}.) 

\begin{lemma}
[Charikar et al.~\cite{CharikarCF2004}]
\label{c-s}
For any $x\in\mathbb R^n$ and $m\ge1$ we have
$|x_i-x^*_i|\le\err_2^m(x)/m^{1/2}$
for all $i\in[n]$ with high probability,
where $x^*$ is the output of the count-sketch algorithm with parameter $m$.
As a consequence we also have
$$\err_2^m(x)\le\|x-\hat x\|_2\le10\err_2^m(x)$$
with high probability, where $\hat x$ is the
$m$-sparse vector best approximating $x^*$ (i.e., $\hat x_i=x^*_i$ for the
$m$ coordinates $i$ with $|x^*_i|$ highest and is $\hat x_i=0$ for the
remaining $n-m$ coordinates).
\end{lemma}

We will also need the following result for the estimation of $L_p$ norms.

\begin{lemma}
[Kane et al.~\cite{KaneNPW2010}]
\label{norm}
For any $p\in(0,2]$ there is a streaming algorithm based on a random linear
map $L:\mathbb R^n\to\mathbb R^l$ with $l=O(\log n)$ that outputs a value $r$
computed solely from $L(x)$ that satisfies
$\|x\|_p\le r\le2\|x\|_p$
with high probability.
\end{lemma}

Our streaming algorithm on Figure~1 makes use of a single count-sketch and two
norm estimate algorithms. The count-sketch is for the randomly scaled version
$z$ of the vector $x$. One of the norm approximation algorithms is for
$\|x\|_p$, the other one approximates $\err_2^m(z)$ through the almost equal
value $\|z-\hat z\|_2$. A standard $L_2$ approximation for $z$ works if we
modify $z$ by subtracting $\hat z$ in the recovery stage. One can get
arbitrary good approximations of $\err_2^m(x)$ this way.

\begin{figure}
\fontsize{9}{11}
  \selectfont
\fbox{
\begin{minipage}[b]{.95\textwidth}

{\bf Initialization Stage:} \\
1. For $0<p<2$, $p\ne1$ set $k=10\lceil1/|p-1|\rceil$ and
$m=O(\epsilon^{-\max(0,p-1)})$ with a large\\
~~~~~~~~~~~~~ enough constant factor.\\
2. For $p=1$ set $k=m=O(\log(1/\epsilon))$ with a large enough constant factor.\\
3. Set $\beta=\epsilon^{1-1/p}$ and $l=O(\log n)$ with a large enough constant factor.\\
4. Select $k$-wise independent uniform scaling factors
$t_i\in[0,1]$ for $i\in[n]$.\\
5. Select the appropriate random linear functions for the execution of the
count-sketch\\ algorithm and $L$ and $L'$ for the norm estimations in the processing stage.\\

{\bf Processing Stage:}\\
1. Use count-sketch with parameter $m$ for the scaled vector $z\in\mathbb R^n$
with $z_i=x_i/t_i^{1/p}$.\\
2. Maintain a linear sketch $L(x)$ as needed for the $L_p$ norm approximation
of $x$.\\
3. Maintain a linear sketch $L'(z)$ as needed for the $L_2$ norm estimation of
$z$.\\

{\bf Recovery Stage:} \\
1. Compute the output $z^*$ of the count-sketch and its best $m$-sparse
approximation $\hat z$.\\
2. Based on $L(x)$ compute a real $r$ with $\|x\|_p\le r\le2\|x\|_p$.\\
3. Based on $L'(z-\hat z)=L'(z)-L'(\hat z)$ compute a real $s$ with $\|z-\hat
z\|_2\le s\le2\|z-\hat z\|_2$.\\
4. Find $i$ with $|z^*_i|$ maximal.\\
5. If $s>\beta m^{1/2}r$ or $|z^*_i|<\epsilon^{-1/p}r$ output FAIL.\\
6. Output $i$ as the sample and $z^*_it_i^{1/p}$ as an approximation for
$x_i$.
\end{minipage}
}
\label{fig:lpsampler}
\caption{
Our approximate $L_p$-sampler with both success probability and relative error $\Theta(\epsilon)$}
\vspace{-5mm}
\end{figure}

First we estimate the probability that the algorithm aborts because $s>\beta
m^{1/2}r$. This depends on the scaling that resulted in $z$ and it will be important
for us that the bound holds even after conditioning on any one scaling factor.

\begin{lemma}\label{abort}
Conditioned on an arbitrary fixed value $t$ of $t_i$ for a single index
$i\in[n]$ we have $\Pr[s>\beta m^{1/2}r\mid t_i=t]=O(\epsilon+n^{-c})$.
\end{lemma}

\begin{proof}
First note that by Lemma~\ref{norm} we have $r\ge\|x\|_p$ and
$s\le2\|z-\hat z\|_2$ with high probability. By
Lemma~\ref{c-s} we have $\|z-\hat z\|\le10\err_2^m(z)$ also
with high probability. We may therefore assume that all of these inequalities
hold, and in particular
$r\ge\|x\|_p$ and $s\le20\err_2^m(z)$. It is therefore enough to bound the
probability that $20\err_2^m(z)>\beta m^{1/2}\|x\|_p$.

For simplicity (and without loss of generality) we assume that the fixed
scalar is $t_n=t$ and will freely use $i$ for indexes in $[n-1]$.

Let $T=\beta\|x\|_p$. For each $i\in[n-1]$ we define
two variables $z'_i$ and $z''_i$ determined by $z_i$ as follows. The indicator
variable $z'_i=1$ if $|z_i|>T$ and $0$ otherwise. We set
$z''_i=z_i^2(1-z'_i)/T^2\in[0,1]$. Let $S'=\sum_{i\in[n-1]}z'_i$ and
$S''=\sum_{i\in[n-1]}z''_i$. Note that $T^2S''=\|z-w\|_2^2$, where $w$ is
defined by $w_i=z_iz'_i$ for $i\in[n-1]$ and $w_n=z_n$. Here $w$ is
$(S'+1)$-sparse, so we have $\err_2^m(z)\le TS''^{1/2}$ unless $S'\ge m$.
It is therefore enough to bound the probabilities of the events
$S'\ge m$ and $S''>m\beta^2\|x\|_p^2/(20T)^2=m/400$, each with  $O(\epsilon)$.

We have $\E[z'_i]=|x_i|^p/T^p$, $\E[S']\le\beta^{-p}=\epsilon^{1-p}$. By our
choice of $m$ and the concentration of $S'$ provided by $k$-wise independence
we have $\Pr[S'\ge m]=O(\epsilon)$ as needed.
The calculation for $S''$ is similar. We have
$$\E[z''_i]<\int_{|x_i|^p/T^p}^\infty x_i^2t^{-2/p}T^{-2}dt=\frac
p{2-p}|x_i|^pT^{-p}.$$ Thus $\E[S'']\le\frac
p{2-p}\|x\|_p^pT^{-p}=O(\beta^{-p})=O(\epsilon^{1-p})$. Note that the $z''_i$
are not indicator variables as the $z'_i$, but they are still $k$-wise
independent random variables from $[0,1]$ and we can bound the probability of
large deviation for $S''$ as we did for $S'$. This completes the proof of the
lemma.
\end{proof}

The fact that our algorithm is an approximate $L_p$-sampler with both relative
error and success probability $\Theta(\epsilon)$ follows from the following
lemma. Indeed, if the probabilities were exactly $\epsilon|x_i|^p/r^p$ and
if $\|x\|_p\le r\le2\|x\|_p$ would always hold, we
would make no relative error and the success probability would be
$\E[\epsilon\|x\|_p^p/r^p]\ge\epsilon/2^p$.

\begin{lemma} \label{lps}
The probability that the algorithm of Figure~1 outputs the index
$i\in[n]$ conditioned on a fixed value for $r\ge\|x\|_p^p$ is
$(\epsilon+O(\epsilon^2))|x_i|^p/r^p+O(n^{-c})$. The
relative error of the estimate for $x_i$ is at most $\epsilon$ with high
probability.
\end{lemma}

\begin{proof} Optimally, we would output $i\in[n]$ if
$|z_i|>\epsilon^{-1/p}r$. This happens if $t_i<\epsilon|x_i|^p/r^p$ and
has probability exactly $\epsilon|x_i|^p/r^p$. We have to estimate the
probability that something goes wrong and the algorithm outputs $i$ when this
simple condition is not met or vice versa.

Three things can go wrong. First, if $s>m^{1/2}\beta r$ the algorithm
fails. This is only a problem for our calculation if it should, in fact,
output the index $i$. Lemma~\ref{abort} bounds the conditional
probability of this happening.

Having dealt with the $s>\beta m^{1/2}r$ case we may assume now $s\le\beta
m^{1/2} r$. We also make the assumptions (high probability by
Lemma~\ref{norm}) that
$\|z-\hat z\|_2\le s$ and thus $\err_2^m(z)\le\|z-\hat z\|_2\le s\le\beta
m^{1/2}r$. Finally, we also assume $|z^*_i-z_i|\le\err_2^m(z)/m^{1/2}\le\beta
r$ for all $i\in[n]$. This is satisfied with high probability by
Lemma~\ref{c-s}.

A second source of error comes from this $\beta r$ possible difference
between $z^*_i$ and $z_i$. This can only make a problem if $t_i$ is
close to the threshold, namely $(\epsilon^{-1/p}+\beta)^{-p}|x_i|^p/r^p\le
t_i\le (\epsilon^{-1/p}-\beta)^{-p}|x_i|^p/r^p$. The probability of selecting
$t_i$ from this interval is
$O(\beta/\epsilon^{1+1/p}|x_i|^p/r^p)=O(\epsilon^2|x_i|^p/r^p)$ as required.

Finally, the third source of error comes from the possibility that $i$ should
be output based on $|z_i|>\epsilon^{-1/p}r$, yet we output another index
$i'\ne i$ because $z^*_{i'}\ge z^*_i$. In this case we
must have $t_{i'}<(\epsilon^{-1/p}-\beta)^{-p}|x_i|^p/r^p$. This has
probability $O(\epsilon|x_{i'}|^p/r^p)$. By the union bound the probability
that such an index $i'$ exists is
$O(\epsilon\|x\|_p^p/r^p)=O(\epsilon)$. Pairwise independence is enough to
conclude that the same bound holds after conditioning on
$|z_i|>\epsilon^{-1/p}r$. This finishes the proof of the first statement of
the lemma.

The algorithm only outputs an index $i$ if $s\le\beta m^{1/2}r$ and
$|z^*_i|\le\epsilon^{-1/p}r$. The first implies that the absolute
approximation error for $z_i$ is at most $\beta r$, while the second lower
bounds the absolute value of the approximation itself by $\epsilon^{-1/p}r$,
thus ensuring a $\beta\epsilon^{1/p}=\epsilon$ relative error
approximation. Our approximation for $x_i=z_it_i^{1/p}$ is $z^*_it^{1/p}$, so
the relative error is the same. Note that the
inverse polynomial error probability comes from the various low probability
events we neglected. The same is true for the additive error term in the
distribution.
\end{proof}

\begin{theorem}\label{thm:sampler} For $\delta>0$ and $\epsilon>0$, $0<p<2$ 
%the
 % $O(\log(1/\delta)/\epsilon))$ repetition of the streaming algorithm in
 % Figure~1 (taking te first non-failing output) 
 there is an $O(\epsilon)$ relative
  error one pass $L_p$-sampling algorithm with failing probability at most $\delta$ and having
  low probability that the relative error of the estimate for the selected
  coordinate is more than $\epsilon$. 
%  
    The algorithm uses
  $O_p(\epsilon^{-\max(1,p)}\log^2n\log(1/\delta))$ space for $p\ne1$ while
  for $p=1$ the space is
  $O(\epsilon^{-1}\log(1/\epsilon)\log^2n\log(1/\delta))$.
\end{theorem}

\begin{proof}
Using Lemma~\ref{lps} and the fact that $\|x\|_p\le r\le2\|x\|_p$ with high
probability one obtains that the failure probability of the algorithm in Figure~1 
is at most $1-\epsilon/2^p+O(n^{-c})$. Conditioning on
obtaining an output, returning $i$ has probability
$(1+O(\epsilon))|x_i|^p/\|x\|_p^p+O(n^{-c})$. Clearly, the latter statement
remains true for any number of repetitions and the failure probability is
raised to power $v$ for $v$ repetitions. Thus using
$v=O(\log(1/\delta)/\epsilon)$ repetitions (taking the first non-failing output),
 the algorithm is an $O(\epsilon)$
relative error $\delta$ failure probability $L_p$-sampling algorithm. Here we  
assume $v<n$ as otherwise recording the entire vector $x$ is more efficient.

The low probability of more than $\epsilon$ relative error in estimating $x_i$
also follows from Lemma~\ref{lps}.
%
In one round, the algorithm on Figure~1 uses
$O(m\log n)$ counters for the count-sketch and this dominates the
counters for the norm estimators. Using standard discretization this can be
turned into an $O(m\log^2n)$ bit algorithm. For the discretization we also
have to keep our scaling factors polynomial in $n$. Recall that in the
continuous model these factors $t_i^{-1/p}$ were unbounded. But we can safely
declare failure if $t_i^{-1}>n^c$ for some $i\in[n]$ as this has low
probability $n^{1-c}$. We have to do the $v$ repetitions of the algorithm
in parallel to obtain a single pass streaming algorithm. This increases the
space to $O(vm\log^2n)$ which is the same as the one claimed in the theorem.
\end{proof}

Note that the hidden constant in the space bound of the theorem depends on
$p$, especially that $1/(2-p)$, $1/p$ and $1/|1-p|$ factors come in. The last
can
always be replaced by a $\log(1/\epsilon)$ factor but the former ones are
harder to handle. For $p=2$ an extra $\log n$ factor seems to be necessary for
an algorithm along these lines, see \cite{AndoniKO2010}.

As we will see in Theorem~\ref{lpl}, our space bound is tight for $\epsilon$
and $\delta$ constants. Note that the lower bound holds also if we only
require the overall distribution of the $L_p$-sampler to be close to the $L_p$
distribution as opposed to the much more strict definition of $\epsilon$
relative error sampling.


% !TeX root = lpsampler.tex

\subsection{The $L_0$ Sampler}
\label{sec:l0samp}
For $p$ near zero, the method of precision sampling becomes
intractable. This is because our scaling factors are
$t_i^{-1/p}$ which clearly rules out $p=0$. In the following we
present a $L_0$ using a different approach. First we state the
following well-known result on exact recovery of sparse vectors.
\begin{lemma}
\label{lem:sparse}
%Let $1\le s\le n$. There is a choice $k=O(s)$ and random linear function
%$L:\mathbb R^n\to\mathbb R^k$ (generated from $O(k\log n)$ random bits) and a
%recovery procedure that on input $L(x)$ outputs $x'\in\mathbb R^n$ or DENSE
%such that for any $s$-sparse vector $x$ the output is $x'=x$ with probability
%$1$ and for any vector $x$ that is not $s$-sparse the output is DENSE with
%probability $1-O(n^{-s})$.
For $1\le s\le n$ and $k=O(s)$ there is a random linear function
$L:\mathbb R^n\to\mathbb R^k$ (generated from $O(k\log n)$ random bits) and a
recovery procedure that on input $L(x)$ outputs $x'\in\mathbb R^n$ or DENSE
such that for any $s$-sparse vector $x$ the output is $x'=x$ with probability
$1$ and for any vector $x$ that is not $s$-sparse the output is DENSE with
high probability.% $1-O(n^{-s})$.
%
%For $1\le s\le n$ and $k=O(s)$ there is a random linear function
%$L:\mathbb R^n\to\mathbb R^k$ (generated from $O(k\log n)$ random bits) and a
%recovery procedure that on input $L(x)$ outputs $x'\in\mathbb R^n$ or DENSE,
%satisfying that for any $s$-sparse $x$ the output is $x'=x$ with
%probability $1$, otherwise the output is DENSE with
%high probability.
%-Mert: Here 'otherwise' does not look right, so I reverted to the previous statement.
\end{lemma}

\begin{theorem}
\label{l0}
There exists a zero relative error $L_0$ sampler which
 uses $O(\log^2 n\log(1/\delta))$ bits and outputs a 
 coordinate $i\in[n]$ with probability at least $1-\delta$.
\end{theorem}
\begin{proof}
We first present our algorithm assuming a random oracle, and
then we remove this assumption through the use of the
pseudo-random generator of Nisan \cite{Nisan1990}. Let $I_k$ for
$k=1,\ldots,\lfloor\log n\rfloor$ be subsets of $[n]$ of size
$2^k$ chosen uniformly at random and $I_0=[n]$. For each $k$ we
run the sparse recovery procedure of Lemma \ref{lem:sparse} on
the vector $x$ restricted to the coordinates in $I_k$ with $s$
set to $\lceil4\log(1/\delta)\rceil$. We return a uniform random
non-zero coordinate from the first recovery that gives a
non-zero $s$-sparse vector. The algorithm fails if each recovery
algorithm returns zero or DENSE.

Let $J$ be the set of coordinates $i$ with $x_i\ne0$ (the
support of $x$). Disregarding the low probability error of the
procedure in \autoref{lem:sparse} this procedure returns each
index $i\in J$ with equal probability and never returns an index
outside $J$. To bound the failure probability we observe that
for $|J|\le s$ failure is not possible, while for $|J|>s$ one
has $k\in[\lfloor\log n\rfloor]$ such that $\E[|I_k\cap
J|]=2^k|J|/n$ is between $s/3$ and $2s/3$. For this $k$ alone
$1\le|I_k\cap J|\le s$ is satisfied with probability at least
$1-\delta$ by the Chernoff bound limiting failure probability by
$\delta$.

  To get rid of the random oracle we use Nisan's generator \cite{Nisan1990} that
  produces the random bits for the algorithm (including the ones describing
  $I_k$ and the ones for the eventual random choice from $I_k\cap J$) from an
  $O(\log^2 n)$ length seed. It fools every logspace tester including the one
  that tests for a fixed set $J\subseteq[n]$ and $i\in[n]$ if the algorithm
  (assuming correct reconstruction) would return $i$. Thus this version of the
  algorithm, now using $O(\log^2n)$ random bits and $O(\log^2\log(1/\delta))$
  total space, is also a zero relative error $L_0$-sampler with failure
  probability bounded by $\delta+O(n^{-c})$.
\end{proof}

As we will see in \autoref{lpl}, this space bound is also tight for
$\delta$ a constant and better sampling is not possible even if we allow
constant relative error or a small overall distance of the output from the
$L_0$ distribution.


% !TeX root = main.tex
\section{Finding Duplicates}
\label{sec:duplicates}
Recall that, given a data stream of length $n+1$ over the alphabet $[n]$, finding
duplicates problem asks to output some $a\in[n]$ that has appeared
at least twice in the stream.

%duplicates problem is defined as outputting an $a\in[n]$ such that $a$ appears
%at least twice in the stream. Observe that by the pigeonhole principle, there
%is always such an $a$. In \cite{Muthukrishnan}, Muthukrishnan asked whether
%finding duplicates problem can be solved using $O(\polylog n)$ space by making
%a constant number of passes over the data. In \cite{Tarui}, Tarui showed that
%any $k$-pass {\em deterministic} algorithm must use $\Omega(n^{1/(2k-1)})$
%space by a reduction from the Karchmer-Wigderson game for the majority
%gate. Finally in  \cite{GopalanJaikumar}, the question posed by Muthukrishnan
%was answered affirmatively by Gopalan et al., who gave a one-pass $O(\log^3
%n)$ space randomized algorithm with constant failure rate. Here we settle the
%one-pass complexity of the problem by giving an $O(\log^2 n)$ space algorithm
%in the next theorem and a $\Omega(\log^2 n)$ lower bound in Theorem \ref{thm:duplb}.

\begin{theorem}\label{thm:dupub}
For any $\delta>0$ there is a $O(\log^2 n\log(1/\delta))$ space one-pass
algorithm which, given a stream of length $n+1$ over the alphabet $[n]$,
outputs an $i\in[n]$ or FAIL, such that the probability of outputting FAIL
is at most $\delta$ and the algorithm outputs a letter $i\in[n]$ that is no
duplicate with low probability.
\end{theorem}
\begin{proof}
Let $x$ be an $n$-dimensional vector, initially zero at each coordinate. We
run the $L_1$-sampler of Theorem~\ref{thm:sampler} on $x$, with both relative error
and failure probability set to $1/2$. Before we start processing the
stream, we subtract 1 from each coordinate of $x$; i.e., we feed the updates
$(i,-1)$ for $i=1,\ldots n$ to the $L_1$ sampling algorithm. When a stream
item $i\in [n]$ comes, we increase $x_i$ by 1; i.e., we generate the update
$(i,1)$.

Observe that when the stream is exhausted, we have $x_i\geq 1$ for items $i$
that have at least two occurrences in the stream, $x_i=0$ for items that
appear once, and $x_i=-1$ for items that do not appear.  Note that our
$L_1$-sampler, if it does not fail, outputs an index $i$ and an approximation
$x^*$ of $x_i$. If $x^*$ is positive, we output $i$, if it is
negative or the $L_1$-sampler fails, we output FAIL. We have
$\sum_{i=1}^nx_i=1$,  hence a perfect $L_1$ sample from $x$ is positive with
more than half probability. Taking into account that our $L_1$-sampler has
$1/2$ relative error and failure probability (and neglecting for a second the
chance that $x^*$ has different sign from $x_i$) we conclude that we output a
duplicate with probability at least $1/4$. The event that $x^*$ does not have
the same sign as $x_i$ (and thus the relative error is at least 1) has low
probability. This low probability can increase the failure probability and/or
introduce error when we output non-duplicate items.

Repeating the algorithm $O(\log(1/\delta))$ times in parallel and accepting
the first non-failing output reduces the failure rate to
$\delta$ but keeps the error rate low.
\end{proof}

As we will see in Theorem~\ref{thm:duplb}, our space bound is tight for
$\delta<1$ a constant. 

It is natural to study the duplicates problem for other ranges of
parameters. Assume that we have a stream of length $n-s\le n$ over the
alphabet $[n]$. For this problem, Gopalan et al.\ \cite{GopalanJaikumar} gave
an $O((s+1)\log^3 n)$ bits algorithm and an $\Omega(s)$ lower bound. Here we
give an algorithm which uses $O(s\log n+\log^2 n)$ space.

\begin{theorem}\label{dups}
For any $\delta>0$ there is an $O(s\log n+\log^2 n\log
  1/\delta)$ space one-pass algorithm which, given a stream of length $n-s$
  over the alphabet $[n]$, outputs NO-DUPLICATE with probability 
  1 if the input sequence has no duplicates, otherwise 
  it outputs $i \in [n]$ or reports FAIL. The returned number is a
  duplicate with high probability while the probability of returning FAIL is at most $\delta$.
\end{theorem}
\begin{proof} Let $x$ be an $n$-dimensional vector updated
  according to the description in the proof of Theorem \ref{thm:dupub}; i.e.,
  $x_i$ is one less than the number of times $i$ appears in the stream. In parallel,
   we run the exact recovery procedure from Lemma \ref{lem:sparse} with parameter $5s$
  and the $1/2$ relative error $L_1$-sampler of 
  Theorem~\ref{thm:sampler} with failure rate $1/2$, both on the vector $x$. If the
  recovery algorithm returns a vector (as opposed to DENSE) we proceed 
  and give the correct output
  assuming we have learned the entire $x$. Otherwise we consider the output of the sampling
  algorithm. If it is $(i,x^*)$ with $x^*>0$ we report $i$ as a duplicate
  otherwise (if $x^*\le0$ or the sampling algorithm fails) we output FAIL.
Define
\begin{align*}
\|x\|^+_1=\sum_{i:x_i>0} |x_i| & &\text{ and }& &\|x\|_1^-=\sum_{i:x_i<0} |x_i|.
\end{align*}
Note that $\|x\|^+_1-\|x\|_1^-=\sum_{i=1}^n x_i = -s$.
 If $\|x\|^+_1 + \|x\|^-_1 \leq 5s$, then $x$ is $5s$-sparse, thus the sparse
recovery procedure outputs $x$ and the algorithm makes no error. Note that the
no repetition case falls into this category. If, however,
$\|x\|^+_1 + \|x\|^-_1 > 5s$, then the probability that a perfect $L_1$ sample
from $x$ is positive is $\|x\|^+_1/\|x\|_1 > 2/5$. Taking into account the
relative error and failing probability (but ignoring the low probability event
of the sampler outputting a wrong sign or sparse recovery algorithm reporting
a vector), we conclude that with probability at
least 1/10 we get a positive sample and a correct output, otherwise we output
FAIL. The failure probability can be decreased to $\delta$ by
$O(\log(1/\delta))$ independent repetitions of the sampler. Note that the
sparse recovery does not have to be repeated as it has low error probability.

The sparse recovery procedure takes $O(s\log n)$ bits by
Lemma~\ref{lem:sparse} for $s>0$ (it takes $O(\log n)$ bits for $s=0$) and
each instance of the $L_1$-sampler requires $O(\log^2 n)$ bits by
Theorem~\ref{lps}, totaling $O(s\log n + \log^2n\log1/\delta)$ bits. 
\end{proof}

Here we do not have a matching lower bound, but only the $\Omega(\log^2n+s)$ that
follows from the $\Omega(s)$ bound in \cite{GopalanJaikumar} and our
$\Omega(\log^2 n)$ bound on the original version of the duplicates problem.

We remark the last two theorems can be stated in a bit more general
form. Instead of considering repetitions in data streams one can consider the
problem of finding an index $i$ with $x_i>0$ for a vector
$x\in\mathbb Z^n$ given by an update stream. Let
$s=-\sum_{i=1}^nx_i$. If $s<0$, then a positive coordinate exists and the
algorithm of Theorem~\ref{thm:dupub} finds one using $O(\log^2 n\log(1/\delta))$
space with low error and at most $\delta$ failure probability.
If $s\ge0$ a positive coordinate does not necessarily exist, but the algorithm
of Theorem~\ref{dups} finds one, report none exists or fails with the error,
failure and space bounds claimed there.

Let us consider finally the version of the duplicates problem, where we have a
stream of length $n+s>n$ over the alphabet $[n]$. Our lower and upper bounds
are even farther in this case. A duplicate can be found using $O(\min\{\log^2n,
(n/s)\log n\})$ bits of memory in one pass with constant probability as
follows. If we sample a random item from the stream, it will appear again
unless that was the last appearance of the letter. As there are at most $n$
last appearances in the stream of length $n+s$, the probability for a uniform
random sample to repeat later is at least $s/(n+s)$. Therefore, if $n/s < \log
n$, we can sample $4\lceil n/s\rceil$ items from the stream uniformly at
random and check if one of them appears again to obtain a constant error
algorithm for finding duplicates. If on the other hand $n/s \geq \log n$, we use
the algorithm in Theorem~\ref{thm:dupub}.

Combining our lower bound for the original version of the duplicates problem 
with the simple lower bound of  $\Omega(\log n)$, we conclude that any 
streaming algorithm that finds a duplicate in length $n+s$ streams must use 
$\Omega(\log^2(n/s)+ \log n)$ bits.



% !TeX root = lpsampler.tex
\section{Lower Bounds}
\label{sec:lb}
All our lower bounds follow from the augmented indexing problem.
This problem is defined as follows. Let $k$ and $m$ be positive
integers. The first player Alice, is given a string $x\in[k]^m$,
while the second player Bob is given an integer $i\in [m]$ and
$x_j$ for $j<i$. Alice sends a single message to Bob and Bob
should output $x_i$.

\begin{lemma}
[Miltersen et al.\ \cite{MiltersenNSW1995}]
\label{lem:ai}
In any one-way protocol in the joint random source model with
success probability at least $1-\delta>\frac{3}{2k}$, Alice must
send a message of size $\Omega((1-\delta)m\log k)$.
\end{lemma}

We will use this lemma by reducing augmented indexing to other
communication or streaming problems.

%%%%%%%%%%%%%%%%%%%%%%%%%%%%%%%%%%%%%%%%
%
%        Subsection: Universal relation
%
%%%%%%%%%%%%%%%%%%%%%%%%%%%%%%%%%%%%%%%%
\subsection{Universal Relation}
\label{sec:ur}
Consider the following two player communication game. Alice gets
a string $x\in\{0,1\}^n$, Bob gets $y\in\{0,1\}^n$ with the
promise that $x\neq y$. The players exchange messages and the
last player to receive a message must output an index $i\in [n]$
such that $x_i\neq y_i$. We call this the {\em universal
relation communication problem} and denote it by $\URn$.

This relation has been studied in detail for deterministic
communication, as it naturally arises in the context of
Karchmer-Wigderson games \cite{KarchmerW1988}. We note however
that our definition is slightly unusual: in most settings both
players must obtain the same index $i$ such that $x_i\neq y_i$,
whereas we are satisfied with the last player to receive a
message learning such an $i$. Clearly, the stronger requirement
can be met in $\lceil\log n\rceil$ additional bits and one
additional round. The additional bits are needed in
deterministic case but we are not concerned with $O(\log n)$
terms for our bounds, so the two models are almost equivalent up
to the shift of one in the number of rounds.

The best deterministic protocol for $\URn$ is due to Tardos and
Zwick~\cite{TardosZ1997}. Improving a previous result by
Karchmer \cite{Karchmer1989}, they gave a 3 round deterministic
protocol using $n+2$ bits of communication with both players
learning the same index $i$ and showed that $n+1$ bits is
necessary for such a protocol. They also gave an $n-\lfloor\log
n\rfloor+2$ bit 2 round deterministic protocol for our weaker
version of the problem, which is also tight except for the $+2$
term. They also gave an $n-\lfloor\log n\rfloor+4$ bit 4 round
protocol, where both players find an index where $x$ and $y$
differ---but not necessarily the same index. This shows that
finding the same difference is harder.

Let $R^k_\delta(U)$ denote the $k$-round $\delta$-error communication
complexity of the communication problem $U$. We write $R_\delta(U)$ to denote
the $\delta$-error communication complexity when the number of rounds is not
bounded. The next proposition follows from similar ideas that were used in 
Theorem~\ref{l0}. See the Appendix for a sketch of the proof.

\begin{proposition}\label{thm:urub}
It holds that $R^1_\delta(\URn)=O(\log^2 n\log\frac{1}{\delta})$ and $R^2_\delta(\URn)=O(\log n\log\frac{1}{\delta})$.
\end{proposition}

\begin{proof}[Proof sketch] One way to deduce the one round protocol is from
  Theorem~\ref{l0}. Alice and Bob run a single pass $L_0$-sampling algorithm
  on $x-y$. This can be achieved by a single message from Alice to Bob
  containing the memory after the first set of updates as in the proof of
  Theorem~\ref{hhl}. The sample $i$ Bob finds is an (almost uniform random)
  index with $x_i\ne y_i$.

Looking more closely to this algorithm we have presented, it finds an index
where $x$ and $y$ disagree from some set $I\subseteq[n]$ that contains at
least one, but not too many such indices. It tries $O(\log n)$ random sets so
that one of them works. One can obtain the two round
protocol by finding such a set in the first round and concentrating on a
single such set in the second round.
\end{proof}

We remark that along similar lines one can find an $O(\log n \log\log n\log1/\delta )$
space two-pass zero relative error $L_0$-sampling algorithm, by estimating  $L_0$ 
of the vector defined by the stream in the first pass using \cite{KaneNW2010}. Next
 we will show that the above proposition is best possible up to the $O(\log\delta^{-1})$
terms. We start with an averaging lemma. The proof can be found in the Appendix.


\begin{lemma}\label{aver} Any protocol for $\URn$ can be turned into one that
outputs every index $i\in[n]$ with $x_i\ne y_i$ with the same probability. The
new protocol uses a joint random source. The number of bits sent, the number
of rounds and the error probability does not change.
\end{lemma}
\begin{proof}
Using the joint random source the players take a uniform 
random permutation $\pi$ of $[n]$ and use it to permute the digits of $x$ and
  $y$. Further they take a uniform random subset $S\subseteq[n]$ and flip the
  digits with coordinates in $S$. This requires no communication. 
Then they run the original protocol on the modified inputs and report
$\pi^{-1}(i)$ if the original protocol reports $i$. 
All indices where $x$ and $y$ differ are reported with equal probability by
symmetry.
\end{proof}

\begin{theorem}\label{thm:urlb}
For any $\delta<1$ constant we have $R^1_\delta(\URn)=\Omega(\log^2 n)$ and
$R_\delta=\Omega(\log n)$.
\end{theorem}

\begin{proof}

Let $P$ be a $\delta$-error protocol for 
$\URn$ and let $\mu$ be the uniform 
distribution on binary string pairs $(x,y)$ 
of Hamming distance 1. Let $p$ be the 
probability, when the inputs are drawn from 
$\mu$, that the protocol terminates on 
Alice's side with a correct answer. Let $q$ 
be the probability, when the inputs are drawn 
from $\mu$, that the protocol terminates on 
Bob's side with a correct answer. Here the 
probabilities are over both the input 
distribution and the shared random string 
$R$. Since the protocol has error probability 
at most $\delta$, we have $p+q\geq 1-\delta$. 
In particular, either $p$ or $q$ is greater 
than $(1-\delta)/2$.

If $q\geq (1-\delta)/2$, we will use $P$ to 
transmit a $\log n$ bit integer from Alice to 
Bob, otherwise we will use $P$ to transmit 
such integer from Bob to Alice. Assume by 
symmetry $q\geq (1-\delta)/2$. Given an 
integer $j\in[n]$, Alice sets $x_j=1$ and 
$x_i=0$ for $i\neq j$. Bob sets $y_i=0$ for 
$i\in[n]$. Then using the shared randomness 
players pick a length $n$ permutation $\pi$ 
and permute their strings according to it. 
Further they pick a length $n$ binary string 
$r$ uniformly at random and XOR the resulting 
strings with $r$. They run protocol $P$ on 
the final strings. Since the final strings 
constructed by the players are distributed 
according to $\mu$, with probability $q$, the 
protocol terminates on Bob's side with a 
correct answer. When this happens the output 
of the protocol is $\pi(i)$, as the strings 
constructed by the players differ only in 
this position. Hence, with probability $q$ 
Bob learns $\pi(i)$ and he can infer $i$ 
as he knows $\pi$.

However, it is a simple fact that to transmit 
an integer from $[n]$ with $q$  probability, 
one needs to send $\Omega(q\log n)$ bits. To 
see this give Alice an integer $J$ chosen 
uniformly at random from $[n]$ and let $\Pi$ 
be the transcript of the protocol, i.e., the 
collection of all messages sent in each 
round. By Fano's inequality 
(Lemma~\ref{lem:fano}), 
\begin{align*}
q\log n - \BEnt(q)\leq 
\I(J:\Pi\emid R)\leq R_\delta(\URn)
\end{align*}
hence, $R_\delta(\URn)\geq \frac{1}{2}
(1-\delta)\log n -1$ as desired.

The second bound comes from considering a uniform random pair $(x,y)$ with
Hamming distance 1. Either player needs to get $\log n$ bits of information to
learn the only index where the strings differ.

To prove the first bound suppose Alice and Bob wants to
solve the augmented indexing problem with Alice receiving $z\in[2^t]^s$ and Bob
getting $i\in [s]$ and $z_j$ for $j<i$.

Let them construct real vectors $u$ and
$v$ as follows. Let $e_q\in\mathbb R^{2^t}$
be the standard unit vector in the direction of coordinate $q$. Alice forms
the vectors $w_j$ by concatenating $2^{s-j}$ copies of $e_{z_j}$, then she
forms $u$ by concatenating these vectors $w_j$ for
$j\in[s]$. The dimension of $u$ is $n=(2^s-1)2^t$. Bob
obtains $v$ by concatenating the same vectors $w_j$ for $j\in[i-1]$ (these are
known to him) and then
concatenating enough zeros to reach the same dimension
$n$.

Now Alice and Bob perform the $R^1_\delta(\URn)$ length
$\delta$ error one round protocol for $\URn$. By Lemma~\ref{aver} we
may assume the protocol returns a uniform random index where $u$ and $v$
differ. Note that each such index reveals one coordinate $z_j\in[2^t]$ to Bob
for $j\ge i$. As $z_j$ is revealed by $2^{s-j}$ such indices more than half the
time when the $\URn$ protocol does not err Bob learns the correct value of
$z_i$. This yields a $R^1_\delta(\URn)$ length one way protocol for augmented
indexing with error probability $(1+\delta)/2$. By Lemma~\ref{lem:ai} we have
$R^1_\delta(\URn)=\Omega(st)$. Choosing $s=t$ proves the theorem.
\end{proof}

\subsection{Finding Duplicates}\label{sec:dublb}
\begin{theorem}\label{thm:duplb}
Any one-pass streaming algorithm that outputs a duplicate with constant
probability  uses $\Omega(\log^2 n)$ space. This remains true even if the
stream is not allowed to have an element repeated more than twice. 
\end{theorem}
\begin{proof}
We show our claim by a reduction from the universal relation. Each of
 Alice and Bob is given a binary string of length $n$, respectively $x$ 
and $y$. Further, the players are guaranteed that $x\neq y$. Alice 
sends a message to Bob, after which Bob must output an index 
$i\in[n]$ such that $x_i\neq y_i$. By Theorem~\ref{thm:urlb}, to solve this 
problem with $1/2$ error probability requires $\Omega(\log^2 n)$ bits for one-way communication. 
Alice constructs the set $S=\{2i-1+x_i\mid i\in[n]\}\subseteq[2n]$ and Bob
constructs $T=\{2i-y_i\mid i\in[n]\}\subseteq[2n]$. Observe that $|S|=|T|=n$ 
and $x_i\neq y_i$ if and only if either $2i$ or $2i-1$ is in both $S$ and
$T$.

Next, using the shared randomness, players pick a random subset 
$P$ of $[2n]$ of size $n$. We have
$$\Pr[|S\cap P| + |T\cap P|\geq n+1]>1/8.$$
To see this let $i\in S\cap T$ and $j\in[2n]\setminus(S\cap T)$. We have
$|P\cap\{i,j\}|=1$ with probability more than $1/2$. The sets $P$ satisfying
this can be partitioned into classes of size four by putting $Q\cup\{i\}$,
$Q\cup\{j\}$ and their complements in the same class for any
$Q\subseteq[2n]\setminus\{i,j\}$, $|Q|=n-1$. Clearly, at least one of the four
sets $P$ in each class satisfies $|S\cap P|+|T\cap P|>n$.

Given a streaming 
algorithm $A$ for finding duplicates, Alice feeds the elements of 
$S\cap P$ to $A$ and sends the memory contents over to Bob, 
along with the integer $|S\cap P|$. If $|S\cap P|+|T\cap P|<n+1$, 
Bob outputs FAIL. Otherwise, feeds arbitrary $n+1-|S\cap P|$ 
elements of $T\cap P$ to $A$. Note that no element repeats more than twice.

On the other hand $|P|=n$ and we always give $n+1$ elements of $P$ 
to the algorithm. Also with constant probability, Bob finds an 
$a\in S \cap T$, which in turn reveals an $i$ such that $x_i\neq y_i$. 
Therefore by \autoref{thm:urlb}, any algorithm for finding 
duplicates must use $\Omega(\log^2 n)$ bits.
\end{proof}

\subsection{$L_p$-sampling}
Our algorithm for the duplicates problem (Theorem~\ref{thm:dupub}) is based on
$L_1$-sampling, thus the matching lower bound for the duplicates problem
implies a similar bound for $L_1$-sampling. Here we show an $\Omega(\log^2 n)$ lower bound for $L_p$-sampling for all $p$.
 Notice that the $L_p$ distribution corresponding to $0,\pm1$
vectors are independent of $p$, so $p$ does not have to be specified for the
next theorem.

\begin{theorem}\label{lpl}
Any one pass
% approximate 
$L_p$-sampler with an output distribution,
whose variation distance from the $L_p$ distribution corresponding to $x$ is
at most $1/3$, requires $\Omega(\log^2n)$ bits of memory. This holds even
when all the coodinates of $x$
%Here $x$ is an $n$
%dimensional integer vector given by an update stream and 
are
guaranteed to be $-1$, $0$ or $1$.

For constants $\delta<1$ and $\epsilon<1$ the same lower bound holds for any
$\epsilon$ relative error $L_p$-sampler with failure probability $\delta$.
\end{theorem}

\begin{proof}
Consider the $L_1$ sampling algorithm that we used to prove
Theorem~\ref{thm:dupub}. Given a stream of items from $[n]$ we turned it to
an update stream for an $n$ dimensional vector $x$ by first
producing an update $(i,-1)$ for all $i\in[n]$ and then for any letter $i$ in
the stream producing an update $(i,1)$. Assuming that no item appears more
than twice in the stream all coordinates of the final vector $x$ are $-1$, $0$
or $1$. The $L_1$ distribution for $x$ puts weight more than $1/2$ on the
coordinates having value $1$. These are the duplicates. Thus if we have
another algorithm such that the variation distance of its output is at most
$1/3$ from this $L_1$ distribution, then it returns a coordinate with value 1
with probability at least $1/6$. For an $\epsilon$ relative error $\delta$
failure probability approximate $L_p$-sampler the same probability is at least
$(1-\epsilon)(1-\delta)-n^{1-c}$. Finding a coordinate in $x$ with value $1$
is the same as finding a duplicate in the original stream, so we need
$\Omega(\log^2n)$ memory by \autoref{thm:duplb}.
\end{proof}



\subsection{Heavy Hitters}\label{sec:hh}
The heavy hitters problem in the streaming model is defined as follows. Let
$x$ be an $n$-dimensional integer vector given by an update stream.
A heavy hitters algorithm with parameters $p>0$ and $\phi>0$ is
required to output a set $S\subseteq [n]$ that contains
all $i$ with $|x_i|\geq \phi\|x\|_p$ and no $i$ such
that $|x_i|\leq \frac{\phi}{2}\|x\|_p$. We call such $S$ a valid heavy hitter
set.\footnote{ In general, the parameter $\frac12\phi$
can be replaced by any $\epsilon < \phi$. 
 Since here our 
focus is on lower bounds, we have simplified the definition.} 
In this part, we show a tight lower bound for the
 space complexity of randomized algorithms (assuming
constant probability of error) for 
the heavy hitter problem. First we briefly review the upper bounds.

The count-median algorithm from \cite{CormodeM2005b} gives
 a $O(\phi^{-1}\log^2 n)$ space upper bound for the case of $p=1$.
Here we note the count-sketch \cite{CharikarCF2004} in fact 
gives a $O(\phi^{-p}\log^2 n)$ space upper bound for all $p \in (0,2]$.
The case of $p=2$ easily follows from Lemma~\ref{c-s}. Let $d=\err^m_2(x)/m^{1/2}$.
In general it holds $d \le ||x||_p/m^{1/p}$ for any
$p\in(0,2]$. Indeed, let $H \subset [n]$ be the set of indices for which
$ d^2=\sum_{i\in H}x_i^2/m$ and let $c=\max_{i\in H}|x_i|$. Then we have
$||x||_p^p/m=\sum_{i\in[n]}|x_i|^p/m\ge c^p+\sum_{i\in H}|x_i|^p/m\ge
c^p+c^{p-2}\sum_{i\in
H}x_i^2/m=c^p+c^{p-2}d^2\ge c^p((1-p/2)+(p/2)c^{-2}d^2\ge
c^p(c^{-2}d^2)^{p/2}=d^p.$ Therefore setting $m=1/\phi^p$ in the count-sketch
scheme gives the desired result.


We remark that a similar upper bound for the heavy hitter problem is 
shown in \cite{KaneNPW2010} (cf.\ Theorem 1), albeit via different arguments.  
 %at the cost of increasing the time to
 %produce the heavy hitter set. 
%For sake of completeness, we briefly describe this modified algorithm in \ref{sec:hhUB}.  
%
%\begin{theorem}[\cite{KaneNPW}, Theorem 1]\label{thm:hhub} Let $p\in (0,2]$ be a real.
% There is a one-pass streaming algorithm that takes 
% $O(\frac{1}{\phi^p}\log(n/\delta)(\log(mM)+\log\log n))$ bits of
%  memory, and outputs a valid heavy hitter set with probability
%   at least $1-\delta$.
%\end{theorem}
%
%In \cite{KaneNPW}, 
In the next theorem, we show that the above upper bound is tight for
any reasonable range of parameters. Our lower bound holds even in the strict
turnstile model and even for very short streams.

\begin{theorem}\label{hhl} Let $p>0$ and $\phi\in(0,1)$ be a reals. Any one
  pass heavy hitter algorithm in the strict turnstile model uses
  $\Omega(\phi^{-p}\log^2n)$.
\end{theorem}

\begin{proof} Suppose there is a one pass heavy hitter algorithm for
  parameters $p$ and $\phi$. We allow for a
  random oracle and assume the updates are polynomially bounded in $n$ and
  integers. We can also restrict the number of updates to be $O(\phi^{-p}\log
  n)$ and assume all coordinates of the final vector are positive (strict
  turnstile model). We turn this streaming algorithm into a protocol for
  augmented indexing in a similar way as we transformed the protocol for
  $\URn$ to a protocol for augmented indexing in the proof of
  Theorem~\ref{thm:urlb}. The exponential growth is now achieved not by
  repetition but by multiplying the coordinates with a growing factor.

Suppose Alice and Bob wants to
solve the augmented indexing problem and Alice receives $y\in[2^t]^s$ and Bob
gets $i\in [s]$ and $y_j$ for $j<i$. Let them construct real vectors $u$ and
$v$ as follows. Let $b=(1-(2\phi)^p)^{-1/p}$ and let $e_q\in\mathbb R^{2^t}$
be the standard unit vector in the direction of coordinate $q$. Alice obtains
$u$ by concatenating the vectors $\lceil b^{s-j}\rceil e_{y_j}$ for
$j\in[s]$. The dimension of $u$ is $n'=s2^t$. Bob
obtains $v$ by concatenating the same vectors for $j\in[i-1]$ and then
concatenating enough zeros, namely $(s-i+1)2^t$, to reach the same dimension
$n'$. Now Alice and Bob perform the heavy hitter algorithm for the vector
$x=u-v$ as follows. Alice generates the necessary updates to increase the
initially zero vector $x\in\mathbb Z^n$ to reach $x=u$, maintains the memory
content throughout these updates and sends the final content to Bob. Now Bob
generates the necessary updates to decrease $x=u$ to its final value $x=u-v$
and maintains the memory throughout. Finally Bob learns the heavy hitter set
$S$ the streaming algorithm produces and outputs $z\in[2^t]$ if the smallest
index in $S$ is $(i-1)2^t+z$.

We claim that the above protocol errs only if the streaming algorithm makes an
error. Notice that all coordinates of $x_l$ of $x=u-v$ are zero except the
ones of the form $x_{l_j}=\lceil b^{s-j}\rceil$ for $l_j=(j-1)2^t+y_j$, where
$i\le j\le s$. Thus $x_{l_i}$ is the first non-zero coordinate. So the claim
is true if $x_{l_i}\ge\phi\|x\|_p$. Using $\lceil v\rceil<2v$ for $v\ge1$ we
get exactly this:
\begin{align*}
\phi^p\|x\|_p^p &= \phi^p\sum_{j=i}^s\lceil b^{s-j}\rceil^p\\
&<(2\phi)^pb^{p(s-i+1)}/(b^p-1)\\
&=b^{p(s-i)} \quad\qquad\text{(since $b^p =1/(1-(2\phi)^p)$)}\\
&\le x_{l_i}^p
\end{align*}

Let us now choose $s=\lceil(2\phi)^{-p}\log n\rceil$ and $t=\lceil\log
n/2\rceil$. For large enough $n$ this gives $n'=s2^t<n$ and all coordinates of
$x$ throughout the procedure remain under $n$. Still if the streaming
algorithm works with probability over 1/2, then  by Lemma~\ref{lem:ai} the message
size of the devised protocol is $\Omega(st)=\Omega(\phi^{-p}\log^2n)$. This
proves the theorem as the message size of the protocol is the same as the
memory size of the streaming algorithm.
\end{proof}


\bibliography{main}{}
\bibliographystyle{alpha}


%%%%%%%%%%%%%%%%%%%%%%%%%%%%%%%%%%%%%%%%
%%%%%%%%%%%%%%%%%%%%%%%%%%%%%%%%%%%%%%%%
%
%        Section: Appendix
%
%%%%%%%%%%%%%%%%%%%%%%%%%%%%%%%%%%%%%%%%
\newpage
\appendix

%\subsection{Space Optimized Heavy Hitters Algorithm}\label{sec:hhUB}
%\begin{theorem}
%[\cite{KaneNPW}]
%\label{thm:hhub}
%Let $x$ be an $n$-dimensional integer vector given by an update stream of $m$
%updates of absolute value at most $M$. There is an $O(\phi^{-p}\log (n/\delta)(\log(mM)+\log\log n))$ space one-pass streaming algorithm which, with probability $1-\delta$, outputs a set $S\subseteq [n]$ such that all $i$ with $|x_i|\geq \phi\|x\|_p$ is in $S$ and $S$ contains no $i$ such that $|x_i|<\frac{\phi}{2}\|x\|_p$.
%\end{theorem}
%\begin{proof}
%Let $\epsilon=\phi^p$, $S = 20\log(n/\delta)$ and $T=64/\epsilon$. Let $h_s: [n]\rightarrow [T]$ be hash functions drawn independently from a 2-wise independent hash family for $s=1,\ldots,S$. We run $ST$ instances of a frequency moment estimation algorithm which gives $1\pm1/16$ approximations to $\|v\|_p^p$ with probability $7/8$, where $v$ are suitably defined sub-vectors of $x$. We refer to each instance of frequency moment estimation algorithm as $D[s,t]$ where $s\in [S]$ and $t\in [T]$.
%The space requirement is $O(\log(mM) + \log\log n)$ per instance, using the algorithm of \cite{KaneNW10}.
%
%Given an update $(i,u)$, we feed it to $D[s,h_{s}(i)]$ for all $s\in [S]$.  Let $C[s,t]$ be the output of the algorithm $D[s,t]$ at the end of the stream. For any $t\in[T]$ and $s\in[S]$, we have
%\begin{align*}
%\E\left[\sum_{i:h_s(i)=t}|x_i|^p\right] = \epsilon \|x\|_p^p/64.
%\end{align*}
%Hence, by an application of Markov's inequality and union bound, with probability 3/4 we have
%\begin{align*}
%(1-1/16)|x_i|^p \leq C[s,t] \leq (1+1/16)(|x_i|^p + \epsilon \|x\|^p_p/8).
%\end{align*}
%Let $e_i$ be the average of $C[s,h_s(i)]$ for $s\in[S]$. By Chernoff bounds, with probability $1-\delta$
%\begin{align*}
%(6/7)|x_i|^p \leq e_i \leq (7/6)(|x_i|^p + \epsilon \|x\|^p_p/8)
%\end{align*}
%for all $i\in [n]$. Hence we can tell apart $i$ for which $|x_i|\geq\phi\|x\|_p$ from $j$ for which $|x_j|<\frac{\phi}{2}\|x\|_p$. The algorithm takes $O(\phi^{-p}\log(n/\delta)(\log(mM)+\log\log n))$ bits as we run $O(\epsilon^{-1}\log(n/\delta))$ instances of the frequency moment estimator of \cite{KaneNW10}.
%\end{proof}
\end{document}
