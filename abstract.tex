% !TeX root = main.tex

\setlength{\abstitleskip}{-\absparindent}
\vspace{-7.5mm}
\begin{abstract}
In this paper, we present near-optimal space bounds for
$L_p$-samplers. Given a stream of updates (additions and
subtraction) to the coordinates of an underlying vector 
$x \in\mathbb R^n$, a perfect $L_p$ sampler outputs the
$i$-th coordinate with probability $|x_i|^p/\|x\|_p^p$ while an approximate
$L_p$ sampler, roughly speaking, outputs the same coordiante with a probability witin the range
$(1\pm \epsilon)|x_i|^p/\|x\|_p^p$.
In SODA 2010, Monemizadeh and Woodruff showed polylog space
upper bounds for approximate $L_p$-samplers and demonstrated
various applications of them.
%
A year later, Andoni, Krauthgamer and Onak improved the
upper bounds and gave a $O(\epsilon^{-p}\log^3 n)$ space
$\epsilon$ relative error and constant failure rate
$L_p$-sampler for $p\in [1,2]$.
%
In this work, we give another such algorithm requiring only
$O(\epsilon^{-p}\log^2n)$ space for $p \in (1,2)$. For 
$p \in (0,1)$, our space bound is $O(\epsilon^{-1}\log^2n)$, 
while for the $p=1$ case we have an 
$O(\log(1/\epsilon)\epsilon^{-1}\log^2n)$ space algorithm.
We also give a $O(\log^2 n)$ bits zero relative error
$L_0$-sampler, improving the $O(\log^3 n)$ bits algorithm due
to Frahling, Indyk and Sohler.

%This considerably improves the previous bound due to
%Monemizadeh et al.

As an application of our samplers, we give better upper bounds
for the problem of finding duplicates in data streams. In case
the length of the stream is longer than the
alphabet size, $L_1$ sampling gives us an $O(\log^2 n)$ space
algorithm, thus improving the previous $O(\log^3 n)$ bound due
to Gopalan and Radhakrishnan.

In the second part of our work, we prove an $\Omega(\log^2 n)$
lower bound for sampling from 0, $\pm1$ vectors 
(in this special case, the parameter $p$ is not relevant for 
$L_p$ sampling). This matches the space of our sampling 
algorithms for constant $\epsilon>0$. We also prove tight space 
lower bounds for the finding duplicates and heavy hitters 
problems. We obtain these lower bounds using reductions from 
the communication complexity problem Augmented Indexing.
\end{abstract}
